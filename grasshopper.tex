\documentclass[12pt]{article}
\usepackage[margin=1in]{geometry}
\geometry{letterpaper}
\usepackage{graphicx}
\usepackage{setspace}
\usepackage{amssymb}
\usepackage{amsmath}
\usepackage{epstopdf}
\usepackage[numbers]{natbib}
\usepackage{authblk}

\title{The grasshopper model: Ecological dynamics far from absorbing states}

\author{A. J. Rominger}

\date{}

\begin{document}
\maketitle

\section{The problem}

How can there be so much grass and so many grasshoppers? Resource
competition models day that resources are limiting to reproduction:
\begin{equation}
  \label{eq:lotka}
  \frac{dN}{dt} = r(1 - N/K)
\end{equation}

What is the meaning of $K$?
\begin{enumerate}
\item $K = \epsilon R$ where $\epsilon$ is the efficiency of an
  organism in turning resource $R$ into offspring; this implies
  entirety of resource is used up before $dN/dt$ is negative.
\item $K = \epsilon \rho R$ where $\epsilon$ is as before and $\rho$
  is the proportion of the biomass of $R$ available to the metabolism
  of the organisms.
\item $K = \epsilon \rho R$ where $\epsilon$ is as before and $\rho$
  is now interpreted as the probability of discovering a packet of
  available resource.
\end{enumerate}

Problems with these interpretations (at least in the case of grass and
grasshoppers):
\begin{enumerate}
\item Resources are abundant
\item All above ground biomass is available to metabolism yet we
  observe these resources are not depleted
\item Resources are easy to find
\end{enumerate}

This implies grasshoppers are far from resource constraints and thus
far from absorbing states (resources depleted and all grasshoppers
extinct). Yet the system is ``stable,'' i.e. there are no local
extirpations of grass or grasshoppers though of course there are
fluctuations in the total biomass of both.  This suggests we need a
new model.

\section{Basics of the grasshopper model}

\begin{enumerate}
\item Death from predation, disease, old age (but \emph{not}
  starvation) are more important than resource limitation
\item satiation means per capita reproductive rate does not increase
  indefinitely w.r.t. resources
\item per capita death rate should be constant w.r.t. resources except
  in extreme case of near starvation conditions
\item predation should include density dependence because pray are
  harder to find when rare
\item in the extreme case where there's almost no resource then
  density dependence should hold for the primary consumers as well,
  i.e. there should be a term (a probability) s.t. $d^2N/dt^2$ is
  negative for very low resource levels or very low probabilities of
  resource discovery
\end{enumerate}

\section{Speciation when resources aren't limiting}

\begin{enumerate}
\item occurs from dispersal limitation and geographic isolation
\item occurs from strong selection on ways to avoid predation/disease
\item occurs from sexual/social mechanisms
\end{enumerate}

Note: all these speciation mechanisms are true for the very polyfagus
grasshoppers along with rare cases of host switching discussed below

\section{Speciation by resource switching when resources aren't
  limiting}

The height of fitness peak is set by metabolic rate of species, not by
amount of resource. Therefore the height of the peak is not relevant to
selecting, rather the gradient of the peak is.

Host switching when resources are not limiting should thus be rare(?)
as is observed in grasshoppers.

Note that if resources are patchy and organisms are dispersal limited
then locally we should see density dependence and apparent resource
competition thus opening up the classic selective mechanisms leading
to speciation by host switching.  The question is: do all clades that
speciate by host switching specialize on rare hosts.  The answer of
course is no.  And in fact these clades would have selective pressure
to eventually switch to common hosts.  Thus it seems more likely that
such clades have metabolic pathways that require them to specialize,
i.e. they have not discovered how to be generalists.  Thus if they
specialize and switch hosts and in the process speciate it is not
necessary (although it is sufficient) for the host to be rare.  There
is the other mechanism that they are constrained to specialize.
Therefore, do these clades climb fitness peaks based the gradient of
the peak as in the case of grasshoppers, and if so what is the
consequence for the predicted rate of speciation if resources are not
limited by specialization is metabolically required?

\end{document}



